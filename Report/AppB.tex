\renewcommand{\baselinestretch}{1.5}
\chapter{Host Interface Code}
\label{App:Host_Software}
\lstset{language=python, numbers=left, showspaces=false,
    showstringspaces=false, tabsize=4, breaklines=true, basicstyle=\tiny, captionpos=b, numbersep=5pt }
\begin{lstlisting}[frame=single,label=listing:IDT]
#!/usr/bin/python
import serial
import fileinput
import sys
import array
print(" \n------Comparison Engine Tester------")

if len(sys.argv) < 4:
	print("Please enter arguments: Filename, Pixel Count, Read Length")
	exit()
print("Requesting Comparison with :", sys.argv[2] , "Pixels, and Read Length of ", sys.argv[3], "\n")
PIXEL_COUNT = int(sys.argv[2])
READ_LENGTH = int(sys.argv[3])

source = open(sys.argv[1],"r")
print("Opened " , source.name)
data_in = []
for line in source:
	data_in.append(line)

a = 0
b = 0
c = 1
d = 0
TOTAL_DATA_COUNT = PIXEL_COUNT*READ_LENGTH
byte_size = int((TOTAL_DATA_COUNT+2)/4)
data_byte_array = bytearray(byte_size)

for i in range(0,TOTAL_DATA_COUNT-1,1):
	if(data_in[a][b] is "A"):
		data_byte_array[d] += 0*c
		#print("FOUND AN A!")
	elif(data_in[a][b] is "T"):
		data_byte_array[d] += 1*c
		#print("FOUND AN T!")	
	elif(data_in[a][b] is "C"):
		data_byte_array[d] += 2*c
		#print("FOUND AN C!")	
	elif(data_in[a][b] is "G"):
		data_byte_array[d] += 3*c
		#print("FOUND AN G!")
	else:
		print("------Found and invalid Character in input file, exiting...")
		exit()
	a += 1
	if a == PIXEL_COUNT : 
		a = 0
		b += 1
	c = c*2
	if c == 16:
		c = 1
		d += 1
#####UART INTERFACE
# print(data_in[4][0])
#
UART= serial.Serial()
UART.baudrate = 115200
UART.port = 'COM3'
UART.timeout = 5;
UART.open()
print("\nSerial Interface Settings: ",UART, "\n")

print("Written ",UART.write(data_byte_array), "Bytes to UART \n")
data_out = UART.read(493)
print("Received response from FPGA \n")

print(data_out)
source.close()
\end{lstlisting}
